%!TEX TS-program = xelatex
%!TEX options = -aux-directory=Debug -shell-escape -file-line-error -interaction=nonstopmode -halt-on-error -synctex=1 "%DOC%"
\documentclass{article}
\input{LaTeX-Submodule/template.tex}

% Additional packages & macros
\usepackage{mdframed}

% Header and footer
\newcommand{\unitName}{High Performance and Parallel Computing}
\newcommand{\unitTime}{Semester 2, 2024}
\newcommand{\unitCoordinator}{Associate Professor Wayne Kelly}
\newcommand{\documentAuthors}{Tarang Janawalkar}

\fancyhead[L]{\unitName}
\fancyhead[R]{\leftmark}
\fancyfoot[C]{\thepage}

% Copyright
\usepackage[
    type={CC},
    modifier={by-nc-sa},
    version={4.0},
    imagewidth={5em},
    hyphenation={raggedright}
]{doclicense}

\date{}

\newenvironment{aside}[1][]
  {\begin{mdframed}[style=0,%
      leftline=false,rightline=false,leftmargin=2em,rightmargin=2em,%
          innerleftmargin=0pt,innerrightmargin=0pt,linewidth=0.75pt,%
      skipabove=7pt,skipbelow=7pt,#1]\small}
  {\end{mdframed}}

\begin{document}
%
\begin{titlepage}
    \vspace*{\fill}
    \begin{center}
        \LARGE{\textbf{\unitName}} \\[0.1in]
        \normalsize{\unitTime} \\[0.2in]
        \normalsize\textit{\unitCoordinator} \\[0.2in]
        \documentAuthors
    \end{center}
    \vspace*{\fill}
    \doclicenseThis
    \thispagestyle{empty}
\end{titlepage}
\newpage
%
\tableofcontents
\newpage
%
\section{Processor Design}
High performance programs must be designed to take advantage of the
hardware they run on. This section discusses how Central Processing
Unit (CPU) design influences how programs are executed, and why
parallelisation is necessary for high performance computing.
\subsection{Von Neumann Architecture}
Most general purpose computers are based on the von Neumann
architecture where instructions and data are kept in the same memory.
In this model, the CPU performs a fetch-decode-execute cycle to execute
instructions, as described below:
\begin{enumerate}
    \item \textbf{Fetch}: The CPU fetches the next instruction from memory into
          the instruction register and increments the program counter.
    \item \textbf{Decode}: The CPU decodes the instruction to determine what
          operation to perform. This may involve one of the following:
          \begin{itemize}
              \item Load data from memory into a register.
              \item Store data from a register into memory.
              \item Perform an arithmetic or logical operation on data
                    in registers.
              \item Transfer control to another part of the program by
                    changing the program counter.
          \end{itemize}
    \item \textbf{Execute}: The CPU performs the operation specified by
          the instruction. This may involve the Arithmetic Logic Unit
          (ALU) for arithmetic and logical operations, or the bus/memory
          module for memory operations.
\end{enumerate}
\subsection{CPU Clock}
Each instruction in the fetch-decode-execute cycle takes a certain
amount of time to complete. This time is determined by the speed of the
CPU which is controlled by a clock that generates a series of pulses at
a fixed frequency. These pulses are called \textbf{clock cycles} and
the time taken for one cycle to complete is called the \textbf{clock
period} of the CPU. This period must be long enough to:
\begin{itemize}
    \item allow electrical signals to propagate through the CPU, and
    \item allow transistors to reach a steady state when switching
          between on and off.
\end{itemize}
An increase in clock frequency therefore results in a faster CPU, but
due to the physical limitations of the speed of light and the size of
transistors inside the CPU, single-core CPU performance has plateaued
over the past decade.

Note in some cases, the CPU may be ``over-clocked'' to run at a higher
frequency than it was designed for, but this can result in incorrect
operation if the physical components are unable to switch states fast
enough, and lead to overheating.
\subsection{Von Neumann Bottleneck}
Main memory resides outside the CPU, and is connected to the CPU via a
shared bus. As the memory controller has a much lower clock frequency
than the CPU, the CPU often stalls while a data operation is being
performed between the CPU and memory. This is known as the \textbf{von
Neumann bottleneck} and is a major limitation of the von Neumann
architecture. To minimise performance loss, memory accesses should
have:
\begin{itemize}
    \item maximal throughout (i.e., transfer as much data as possible
          in one operation), and
    \item minimal latency (i.e., minimise the time taken to start and
          end an operation).
\end{itemize}
\subsubsection{Cache Memory}
A common way CPUs achieve this is by using \textbf{cache memory} to
store frequently accessed data and instructions in small specialised
blocks of memory that are close to each core.\ These caches provide
higher throughput and lower latency than main memory, but are typically
more expensive and therefore have limited capacity. A CPU often has
multiple levels of cache, starting with the Level 1 (L1) cache which is
the smallest and fastest, increasing in size and latency as the level
increases. On modern CPUs, the L1 cache is typically split into two
parts: one for \textit{instructions} and one for \textit{data}.
Multi-core CPUs may also have unified L2 and L3 caches that are shared
between cores.

\textbf{Cache Operation}\quad The cache works by loading data in blocks of a fixed size called
\textbf{cache lines}. When the CPU requests less than a cache line of
data, it loads surrounding data into the cache to take advantage of
locality of reference. This is done in the hope that the CPU will
eventually request the surrounding data, which will already be in the
cache and is therefore faster to access. When memory is requested, the
CPU first checks if it exists in the cache, requesting it from main
memory only if it is not found.

\textbf{Cache Replacement}\quad When the cache is full, the CPU must decide what data to replace in the
cache, through a \textit{cache replacement policy}. Typically this is
done using the Least Recently Used (LRU) policy, where the data that
has not been accessed for the longest time is replaced. The
\textit{cache placement policy} determines where new data is placed,
and can be:
\begin{itemize}
    \item \textbf{Fully associative}, where any cache line can be replaced.
    \item \textbf{N-way set associative}, where the cache is divided
          into N sets, and the replacement policy is applied within each set.
    \item \textbf{Direct mapped}, where new data must be mapped to a
          specific cache line.
\end{itemize}
\subsection{Instruction-Level Parallelism}
Scalar processors are designed to process and execute instructions one
at a time. High-performance processors take advantage of disjoint
operations that can be performed in parallel through
\textbf{instruction level parallelism} (ILP). This can be achieved in
several ways:
\begin{itemize}
    \item \textbf{Superscalar Execution}: Dispatching multiple
          scalar operations to different functional units (ALU, integer
          multiplication, FPU, store/load, etc.) during a single
          clock cycle. This is facilitated by hardware that determines
          dependencies between instructions and schedules them for
          parallel execution.
    \item \textbf{Out-of-Order Execution}: Executing other instructions
          while waiting for data required by the current instruction.
          As this can result in instructions being executed out of
          order, the processor must ensure that the final result is
          correct.
    \item \textbf{Instruction Pipelining}: Dividing instruction
          processing into multiple stages to keep the processor busy.
          For example the fetch-decode-execute cycle can be divided
          into three stages, with each stage being executed in
          parallel, allowing the processor to begin executing the
          fetch stage of the next instruction while the decode stage
          of the current instruction is being executed.
\end{itemize}
To keep the processor busy, processors may also employ \textbf{branch
    prediction} to predict the outcome of branch instructions and
\textbf{speculative execution} to execute instructions that may not be
required, but are likely to be executed.

Each of these techniques result more complex processor cores that
occupy more space on the CPU chip, consume more power, and have longer
cycle times.
\subsection{Processor Architectures}
There are two main types of processor architectures:
\begin{itemize}
    \item \textbf{Complex Instruction Set Computing (CISC)}: Designed
          to execute complex instructions that can perform multiple
          operations in a single instruction. These instructions may be
          decoded into micro-operations that are executed by the CPU.
          These processors have large instruction sets and typically
          have slower clock speeds.
    \item \textbf{Reduced Instruction Set Computing (RISC)}: Designed
          to execute simple instructions that perform a single
          operation. These processors have smaller instruction sets and
          typically have faster clock speeds.
\end{itemize}
\subsection{Multi-Core Processors}
\begin{aside}[frametitle={Moore's Law}]
    Moore's Law states that the number of transistors on a CPU chip doubles
    approximately every two years. This has led to:
    \begin{itemize}
        \item Smaller circuits
        \item Faster clock speeds
        \item Additional complex hardware level optimisations
    \end{itemize}
    While this has held true for the past few decades, it is now
    increasingly difficult to reduce the size of transistors due to physical
    limitations, such as the size of a wire compared to the width of an atom.
\end{aside}
Rather than allocating this additional space to a single core, CPUs now
contain multiple cores that can execute an independent instruction
stream\footnote{Note that each core may execute the same stream if each
    stream operates on different data, as in Graphical Processing Units.}.
Each core is itself a complete processor, with its own functional units,
program counter, instruction register, general-purpose registers, and
cache.

These cores can communicate with each other via shared memory, but this
leads to the same bottleneck as before. To overcome this, cores may
have their own private caches, and communicate with each other via a
shared L2 or L3 cache.
\subsubsection{Cache Coherence}
When multiple cores share the same memory, it is important to ensure
that each core has the most up-to-date copy of the data. Cache
coherence is the consistency of data stored in multiple private caches
that reference shared memory. Cache mechanisms are used to ensure that
modifications to data in private caches are propagated to other cores
to ensure other cores do not read stale data. This is typically done
using \textit{snooping} or \textit{directory-based} mechanisms:
\begin{itemize}
    \item \textbf{Snooping}: Each cache monitors the bus for
          access to memory locations that have been cached. If a write
          is detected, the cache either updates its data or invalidates
          it.
    \item \textbf{Directory-based}: A centralised directory keeps track
          of data shared between cores. Each processor must request
          permission to load an entry from a memory location. When an
          entry is updated, the directory either updates or invalidates
          other caches with that entry.
\end{itemize}
Caches also have inclusion/exclusion policies that determine whether
data in a lower-level cache may be present in a higher-level cache.
\subsection{Software Parallelism}
Parallelism can be achieved at the software level through processes and
threads. A \textbf{process} is an instance of a program being executed
by the operating system that has its own memory space and threads of
execution. Upon initialisation, a process creates a \textbf{thread}
that executes its main function. This main thread can then spawn
additional threads that execute independent streams of instructions.

A \textbf{thread} is a lightweight process that has its own program
counter and local memory called the \textbf{stack}. The lifetime of the
data in the stack is determined by the function that created it.

Every process also has a shared memory space called the \textbf{heap},
where data with indeterminate lifetimes is stored (i.e., dynamically
allocated objects). This space can be accessed by threads within the
same process, and by threads in other processes through inter-process
communication APIs.
\subsubsection{Mapping Threads to Cores}
Operating systems achieve concurrency in threads through time slicing,
where each thread is allocated a core for a fixed amount of time. When
a time slice ends, the operating system performs a context switch to
another thread by saving the current thread's state and loading the
next thread's state. This allows multiple threads to make progress on a
single core, and allows threads to be executed in parallel on systems
with multiple cores. Note that for a CPU with N cores, a process with N
threads may not always be granted all N cores, as the operating system
may allocate some cores to other processes.
\subsection{Simultaneous Multi-Threading}
To reduce the overhead incurred during a context switch, some
processors allow more than one thread to be executed on a single core
at the same time by interleaving instructions from multiple threads.
This is known as \textbf{Simultaneous Multi-Threading (SMT)} or
\textbf{Hyper-Threading}. In these processors, each core has multiple
sets of registers and program counters (typically two) for each thread,
with a shared set of functional units, that allow multiple threads to
be executed in parallel.
\section{Parallel Computing}
\subsection{Forms of Parallelism}
Parallelism can be achieved at multiple levels:
\begin{itemize}
    \item \textbf{Instruction Level Parallelism (ILP)}: Parallelism
          achieved by executing multiple instructions in parallel.
    \item \textbf{Vector Parallelism}: Parallelism achieved by
          executing the same instruction on multiple data elements
          simultaneously. This is typically done using \textbf{SIMD}
          (Single Instruction, Multiple Data) instructions.
    \item \textbf{Thread Level Parallelism (TLP)}: Parallelism achieved
          by executing multiple threads in parallel. This can be done
          using multiple cores, or by interleaving instructions from
          multiple threads on a single core.
    \item \textbf{Process Level Parallelism}: Parallelism achieved
          by executing multiple processes in parallel. Processes can
          communicate with each other through inter-process
          communication. Threads within the same process communicate via
          shared memory, while processes on different machines
          communicate via message passing. Distributed shared memory
          systems provide a shared memory programming model for
          distributed memory systems.
\end{itemize}
\subsection{Types of Parallel Computers}
\subsubsection{Flynn's Taxonomy}
Flynn's Taxonomy classifies parallel computers based on the number of
instruction streams and data streams that can be processed at the same
time. It has four categories:
\begin{itemize}
    \item \textbf{Single Instruction, Single Data (SISD)}: A single
          instruction stream is executed on a single data stream. This
          is the traditional von Neumann architecture.
    \item \textbf{Single Instruction, Multiple Data (SIMD)}: A single
          instruction stream is executed on multiple data streams
          (includes superscalar processors). This is typically done
          using vector processors or GPUs.
    \item \textbf{Multiple Instruction, Single Data (MISD)}: Multiple
          instruction streams are executed on a single data stream.
          Commonly used in fault-tolerant redundant systems.
    \item \textbf{Multiple Instruction, Multiple Data (MIMD)}: Multiple
          instruction streams are executed on multiple data streams.
          This is the most common form of parallelism, and is used in
          multi-core CPUs and distributed systems, including both
          shared memory and distributed memory systems.
\end{itemize}
\subsection{Types of Supercomputers}
\begin{itemize}
    \item \textbf{Vector Processors}: Processors that can execute
          vector instructions on multiple data elements in parallel.
          This includes early supercomputers like the Cray-1.
    \item \textbf{Symmetric Multi-Processor (SMP)}: Multi-core
          processors that share memory and have equal access to all
          resources.
    \item \textbf{Massively Parallel Processors (MPP)}: Tightly coupled
          systems with multiple processors that communicate with each
          other via proprietary high-speed interconnect.
    \item \textbf{Clusters}: Loosely coupled distributed memory systems
          that consist of commodity nodes.
    \item \textbf{Asymmetric Multi-Processor (AMP)}: Specialised
          co-processors that offload specific tasks from the main CPU (i.e., GPUs).
    \item \textbf{Hybrid Systems}: Systems that combine multiple
          architectures to take advantage of the strengths of each
          architecture.
    \item \textbf{Cycle Stealing Systems}: Systems that use idle
          resources on a network of computers to perform computations.
\end{itemize}
\subsection{Parallel Computing Models}
\subsubsection{Concurrent Computing}
A concurrent computing model is one where multiple operation streams
progress independently. This does not require multiple processors nor
does it imply simultaneous execution. Such models may be prone to
problems such as deadlocks.
\subsubsection{Parallel Computing}
A parallel computing model is one where multiple operation streams
progress simultaneously. This requires multiple processors and
simultaneous execution.
\subsubsection{Distributed Computing}
A distributed computing model is one where multiple operation streams
progress simultaneously on different machines. This includes parallel
computing clusters and distributed concurrent systems.
\subsection{Parallelisation}
Parallelisation is the process of converting a sequential program into
a parallel program, that can run on parallel hardware. Doing this
effectively requires programmers to use a fundamentally different
algorithm than the one used on sequential hardware, that is expressed
in a manner that makes parallelisation more explicit or easier.

In some cases, we can exploit parallelism without changing the
algorithm, by analysing which computational steps performed in the
algorithm can be executed in parallel. This is known as exploiting
\textbf{inherent parallelism}.
\subsubsection{Parallelisation in Functional and Imperative Programming}
\begin{itemize}
    \item \textbf{Functional Programming}: Pure functional programming
          languages express computation via function evaluation. As
          function evaluation does not produce side effects, functions
          that do not depend on the result of another can be executed in
          parallel.
    \item \textbf{Imperative Programming}: Imperative programming
          languages express computation via a sequence of statements.
          Parallelisation in imperative programming is more difficult
          as statements may have side effects that depend on the order
          of execution.
\end{itemize}
\subsection{Safe Parallelisation}
The process of parallelisation must be done in a manner that ensures
the same result is produced as the sequential program. Two dependencies
must be preserved to ensure safe parallelisation:
\begin{itemize}
    \item Control dependencies
    \item Data dependencies
\end{itemize}
\subsubsection{Control Dependences}
A control dependency is a dependency where one statement depends on
whether another statement is executed. Some examples of control
dependencies are shown below:
\begin{minted}{c}
int gcd(int a, int b) {
    // if statement
    if (b == 0) {
        return a;             // Control dependency
    } else {
        return gcd(b, a % b); // Control dependency
    }
}

int gcd(int a, int b) {
    // while loop
    while (b != 0) {
        int temp = b; // Control dependency
        b = a % b;    // Control dependency
        a = temp;     // Control dependency
    }
    return a;
}

int gcd(int a, int b) {
    // error handling
    if (a < 0 || b < 0) {
        return -1;
    }

    ... // Control dependency
}
\end{minted}
Control dependencies can be difficult to determine when the program
contains many conditional statements as the control flow depends on
input data.
\subsubsection{Data Dependencies}
A data dependency is a dependency where one statement depends on the
same data as another statement. Data dependencies can be further
classified into:
\begin{itemize}
    \item \textbf{True Dependencies (W to R)}: Where a
          statement depends on the result of a previous
          instruction.
          \begin{minted}{c}
a = 1;
b = a; // True dependency on statement 1
c = b; // True dependency on statement 1 and 2
\end{minted}
    \item \textbf{Anti-Dependence (R to W)}: Where a
          statement requires a value that is later written to.
          \begin{minted}{c}
a = 1;
b = a; // Anti-dependency on statement 3
a = 2;
\end{minted}
    \item \textbf{Output Dependence (W to W)}: Where a
          variable depends on the ordering of another write
          statement.
          \begin{minted}{c}
// Output dependency between statements 1 and 3
a = 1;
b = a;
a = 2;
\end{minted}
    \item \textbf{Input Dependence (R to R)}: Where a
          statement reads
          \begin{minted}{c}
a = 1;
b = a; // Input dependence on statement 3
c = a; // Input dependence on statement 2
\end{minted}
          Note that the order of input dependencies do not need to be
          preserved, as they do not affect the result of the program.
\end{itemize}
Data dependencies can be more challenging to determine when a program
introduces pointers or references to objects, as such aliases may
inadvertently introduce dependencies between statements. For example:
\begin{minted}{c}
T a = T::new();
T *b = &a;       // b is a reference to a

a.mutator_method();  // changes a
b->mutator_method(); // also changes a
\end{minted}
In this example, some programming languages may not explicitly show the
true dependency of the final statement.
\subsubsection{Dependency Analysis}
Dependency analysis allows us to determine whether it is safe to
reorder or parallelise statements in a program. For example, given the
following code:
\begin{minted}{c}
for (int i = 0; i < n; i++) {
    for (int j = 0; j < n; j++) {
        a[i][j + 1] = a[n][j];
    }
}
\end{minted}
we may wish to know whether there are any data dependencies between
loop iterations. That is, does there exist an iteration \(\left( i_r,\:
j_r \right)\) that reads the same array element that is written to by
iteration \(\left( i_w,\: j_w \right)\)? Mathematically,
\begin{align*}
    \exists i_r,\: j_r,\: i_w,\: j_w : {} & 0 \leqslant i_r < n \land 0 \leqslant j_r < n \land {} \\
                                          & 0 \leqslant i_w < n \land 0 \leqslant j_w < n \land {} \\
                                          & i_w = n \land j_w + 1 = j_r.
\end{align*}
Another way to analyse a loop would be to draw an iteration space
diagram, with one axis representing the iteration number and the other
representing the array index. This allows us to visualise the
dependencies between iterations.

Unfortunately, any form of static dependency analysis is inexact in
general.
\begin{itemize}
    \item Alias analysis is undecidable
    \item Data dependence analysis are undecidable
\end{itemize}
When in doubt, we must assume that a dependency exists, and therefore
cannot parallelise the code.
\subsection{Automatic Parallelisation}
Automatic parallelisation can be performed either
\begin{itemize}
    \item Automatically by the compiler, or
    \item Manually by the programmer.
\end{itemize}
Current compilers are not smart enough to perform parallelisation in
general as they are necessarily conservative in their analysis. On the
other hand, manual parallelisation requires competence and can be very
time-consuming and error-prone.
\subsection{Parallelisation Process}
Generally, we can break the process of parallelisation into three
steps:
\begin{enumerate}
    \item Determine what can be safely parallelised. This may involve:
          \begin{itemize}
              \item Analysing control and data dependencies.
              \item Transforming the program to expose parallelism.
              \item Employing a more efficient algorithm that is better
                    suited for parallelism.
          \end{itemize}
    \item Decide if parallelism will increase performance:
          \begin{itemize}
              \item Is a significant amount of time spent in that code?
              \item Is there a overhead associated with creating and
                    managing threads?
              \item Will the latency and/or throughput of communication
                    between processors be a bottleneck?
              \item Does parallelisation actually result in a speedup?
          \end{itemize}
    \item Transform the program into an explicitly parallel form:
          \begin{itemize}
              \item Use programming language constructs to map
                    computation (and possibly data) to processors and
                    appropriate synchronisation.
          \end{itemize}
\end{enumerate}
\subsection{Parallelism Granularity}
While a potential to parallelise code may exist, executing this code
has an associated overhead due to the costs of creating and managing
threads, the synchronisation of parallel computations, and the latency
of messages sent between processors. Therefore the amount of
computation in each ``work unit'' must be sufficiently large to offset
this overhead.

Here we can introduce the concept of \textbf{parallelism granularity},
which is the size of the work unit that is executed in parallel.
Generally,
\begin{itemize}
    \item \textbf{Coarse-grained parallelism} has large work units that
          are executed in parallel. Distributed systems and clusters
          are suited for coarse-grained parallelism.
    \item \textbf{Fine-grained parallelism} has small work units that
          are executed in parallel. CPUs with vector computation units
          are suited for fine-grained parallelism.
\end{itemize}
\subsection{Speedup}
The speedup of a parallel program is the ratio of the time taken to
execute the best sequential program to the time taken to execute the
parallel program:
\begin{equation}
    S = \frac{\text{execution time of best sequential program}}{\text{execution time of parallel program}}
\end{equation}
Note that the best sequential program is not the same as the parallel
program restricted to a single processor, due to the aforementioned
overheads. Rather, it is the best possible sequential program that can
be written for the given problem, which may use different algorithms.
\subsubsection{Speedup Curves}
Commonly, we will plot the speedup of a parallel program against the
number of processors used. The resulting curves can be classified into
five categories:
\begin{itemize}
    \item \textbf{Super-Linear Speedup}: The speedup increases faster
          than the number of processors used. This typically only occurs
          in extreme cases where the parallel program runs on different
          hardware that allows it to outperform the sequential program.
    \item \textbf{Linear Speedup}: The speedup increases linearly with
          the number of processors used. This is the ideal case, but is
          rarely achieved due to overheads associated with parallelism.
    \item \textbf{Sub-Linear Speedup}: The speedup increases slower than
          the number of processors used. This is the most common case,
          and often the speedup plateaus as the number of processors
          increases.
    \item \textbf{No Speedup}: The speedup remains constant as the
          number of processors used increases. This indicates that the
          program likely is not parallelisable.
    \item \textbf{Slowdown}: The speedup decreases below 1 as the number
          of processors used increases. This indicates that the overhead
          of parallelism is greater than the benefits of parallelism.
\end{itemize}
\subsubsection{Scalable Parallelism}
Typically an increase in problem size results in an increase in total
execution time. In such cases, we want large problems to provide more
potential for parallelism. A problem is said to be \textbf{scalable} if
the speedup of the parallel program increases with the problem size,
without plateauing.
\subsubsection{Computational Complexity}
While we may be able to achieve scalable parallelism, it is important
to note that the computational complexity of the problem does not
change. This is because the computational complexity of a problem is
determined by the size of the problem, and distributing the problem
across finitely many processors does not change the limiting behaviour
of the problem.
\begin{equation*}
    \mathcal{O}\left( n \log{\left( n \right)} / c \right) = \mathcal{O}\left( n \log{\left( n \right)} \right)
\end{equation*}
(\(n\) is the size of the problem and \(c\) is the number of
processors used).
\subsubsection{Amdahl's Law}
Amdahl's Law states that the speedup of a parallel program is limited
by the fraction of the program that cannot be parallelised. If \(p\) is
the fraction of the program that can be parallelised, then the
theoretical speedup \(S\) of the execution of the program is given by:
\begin{equation}
    S = \frac{1}{\left( 1 - p \right) + p/s}
\end{equation}
where \(s\) is the speedup of the parallelisable portion of the program.
\subsection{Parallelisation Methodology}
When we want to parallelise a program, we can take the following steps:
\begin{enumerate}
    \item Obtain representative and realistic data sets.
    \item Time and profile the sequential version of the program.
    \item View the source code and understand the high-level structure
          of the program.
    \item Analyse dependencies.
    \item Determine sections that can be parallelised.
    \item Decide what parallelism might be worth exploiting.
    \item Consider restructuring the program or replacing algorithms to
          expose more parallelism.
    \item Transform the program into an explicitly parallel form.
    \item Test and debug the parallel version of the program.
    \item Time and profile the parallel version of the program.
    \item Determine issues inhibiting greater performance.
\end{enumerate}
\subsubsection{Timing and Profiling}
When timing an application, we must:
\begin{itemize}
    \item Time a variety of data sets.
    \item Time multiple times to get an average.
    \item Be aware of other applications running in the background.
    \item Compare performance under identical conditions.
    \item Eliminate any unnecessary screen I/O by redirecting outputs
          to a file.
    \item Use high resolution timers.
    \item Distinguish between timing the entire program and timing
          specific sections.
    \item Distinguish between real time (wall clock time) and CPU time
          (time actually spent executing instructions in the program).
    \item Be aware of one-off costs (e.g., just-in-time
          compilation/dynamic library loading).
\end{itemize}
Profiling is the process of analysing the performance of a program by
measuring the time spent in each section of the program. Profiling can
capture time spent executing a section of code and the number of times
that section is executed. Profiling can be done through sampling where
the program counter is probed at regular intervals, or through
instrumentation where the program is modified to record the time spent
in each section. It is important to note that profiling can introduce
overheads that may affect the performance of the program. Additionally,
some compilers may allow the programmer to optimise a program aggressively
so that the profiled program may not be representative of the actual
program (i.e., sections may be optimised out or reordered).
\subsection{Premature Optimisation}
\begin{aside}
    Roughly 80\% of effects come from 20\% of causes.

    \raggedleft{\textbf{Pareto Principle}}
\end{aside}
\begin{quote}
    ``There is no doubt that the grail of efficiency leads to abuse.
    Programmers waste enormous amounts of time thinking about, or worrying
    about, the speed of noncritical parts of their programs, and these
    attempts at efficiency actually have a strong negative impact when
    debugging and maintenance are considered. We should forget about small
    efficiencies, say about 97\% of the time: premature optimization is the
    root of all evil.

    Yet we should not pass up our opportunities in that critical 3\%. A
    good programmer will not be lulled into complacency by such
    reasoning, he will be wise to look carefully at the critical code;
    but only after that code has been identified. It is often a mistake
    to make a priori judgments about what parts of a program are really
    critical, since the universal experience of programmers who have
    been using measurement tools has been that their intuitive guesses
    failed''.

    \raggedleft{\textbf{Donald Knuth}}
\end{quote}
\subsection{Explicit Parallelism}
To transform a program into an explicitly parallel form, we must map
computation and data to processors and synchronise processors at
appropriate times. The exact mechanism for this depends on the
programming model or parallelisation framework.
\subsubsection{Parallel Programming Models}
There are two ways of abstracting the underlying hardware that is used
in parallel computing:
\begin{itemize}
    \item \textbf{Communicating Sequential Processes (CSP)}
          \begin{itemize}
              \item Processes operate independently and interact with
                    each other solely through message-passing.
              \item Both computation and data to processors.
              \item Used by most distributed memory machines.
          \end{itemize}
    \item \textbf{Parallel Random Access Machine (PRAM)}
          \begin{itemize}
              \item Processes operate asynchronously and have
                    constant-time access to shared memory.
              \item Only computation needs to be mapped to processors.
              \item Used for thread level parallelism.
          \end{itemize}
\end{itemize}
\subsection{Mapping Computation to Processors}
When mapping threads to processors, they can either be created
explicitly or implicitly:
\begin{itemize}
    \item Explicit threads
          \begin{itemize}
              \item Created by the programmer in code using a function
                    call.
              \item Usually associated with a function that is executed
                    when the thread is first started.
              \item Threads are explicitly started, stopped, or paused.
          \end{itemize}
    \item Implicit threads
          \begin{itemize}
              \item Created by a high-level library that builds on top
                    of primitive threading libraries.
              \item The programmer must only indicate what calculations
                    can be performed in parallel.
              \item Thread creation, management, and synchronisation is
                    handled by the library.
          \end{itemize}
\end{itemize}
\subsubsection{Thread Scheduling}
Threads are mapped to processors by the \textbf{scheduler} which is
part of the operating system. This allows different threads to be
mapped to different processors, and allows situations with more threads
than processors to be handled through interleaving. Threads can be
assigned a relative \textbf{priority} which the scheduler uses to
determine which thread to execute next. Threads can also be assigned a
\textbf{processor affinity} to ensure that they are always executed on
the same processor. This can be useful when we want to reuse local
caches.

There are several scheduling algorithms that the operating system can
use to determine which thread to execute next:
\begin{itemize}
    \item \textbf{Pre-emptive}: The scheduler can interrupt a thread
          that is currently executing to run another thread of a higher
          priority.
    \item \textbf{Cooperative}: The scheduler relies on threads to
          voluntarily yield control to other threads.
    \item \textbf{Round Robin}: Each thread is given a fixed time slice
          to execute before the next thread is executed.
    \item \textbf{First-Come, First-Served (FCFS)}: Threads are executed
          in the order they are created.
    \item \textbf{Shortest Job First (SJF)}: Threads are executed in
          order of the time they are expected to take to complete.
\end{itemize}
\subsubsection{Thread Pools}
As creating new threads is an expensive operation, we can instead
create a pool of threads at the start of a program to avoid needing to
create and destroy threads each time we want to parallelise a section
of code. When threads are needed, they are taken from the pool, and
when they are done, they are returned to the pool. New threads can also
be added to the pool when necessary. Threaded applications can either
use thread pools explicitly or use a high-level library that implements
thread pools internally.
\subsubsection{Load Balancing} When mapping work to processors, it is important to balance the work
done by each processor. This can be done using
\begin{itemize}
    \item \textbf{Dynamic work allocation}: where work is overdecomposed
          into many work units, typically far more than the number of
          processors. This is good if execution times are unpredictable
          or if the work is not evenly distributed, but can have
          larger overheads. Another approach can involve creating work
          pools where we have a queue of tasks awaiting execution. Here
          work can be centralised or decentralised and work can be
          stolen from other processors if a processor runs out of work.
    \item \textbf{Static work allocation}: where work is divided into
          equal-sized chunks and assigned to each processor. This is
          good if execution times are predictable and work is evenly
          distributed, but can lead to processors idling if work is not
          evenly distributed. Work can be divided:
          \begin{itemize}
              \item \textbf{By blocks}: which offer good locality
                    but may suffer from poor load balance if work is not
                    evenly distributed.
              \item \textbf{Cyclically}: where work is assigned in a
                    round-robin fashion. This can be good if work is evenly
                    distributed, but can lead to poor locality.
          \end{itemize}
\end{itemize}
\section{Synchronisation}
Some parallel programs require threads to be executed in a particular
order to ensure access to shared memory is correct.
\subsection{Race Conditions}
Consider the problem of adding the elements in an array.
\begin{minted}{c}
#define N 100

int array[N] = {0};
int sum = 0;

for (int i = 0; i < NUM_THREADS; i++) {
    thread_create(add, i);
}

void add(int thread_id)
{
    int block_size = ARRAY_SIZE / NUM_THREADS;
    int start = thread_id * block_size;

    if (start + block_size < N)
    {
        for (int i = start; i < start + block_size; i++)
        {
            sum += array[i];
        }
    }
    else
    {
        for (int i = start; i < N; i++)
        {
            sum += array[i];
        }
    }
}
\end{minted}
Assume at some point in time, the value of \mintinline{c}{sum} is
\(s\), and let two threads be scheduled to execute the
\mintinline{c}{add} function at indices \(i_1 = 20\) and \(i_2 = 30\),
where \mintinline{c}{array} has the values 20 and 30. Now note that the
compound operation being performed within the \mintinline{c}{add}
function is not atomic, and will be expanded to the following
operations for thread 1:
\begin{enumerate}
    \item Read \mintinline{c}{sum}.
    \item Read \mintinline{c}{array} at index \(i_1\).
    \item Add \mintinline{c}{sum} and \mintinline{c}{array[i1]}
    \item Write result to \mintinline{c}{sum}.
\end{enumerate}
and similarly for thread 2:
\begin{enumerate}
    \item Read \mintinline{c}{sum}.
    \item Read \mintinline{c}{array[i2]}
    \item Add \mintinline{c}{sum} and \mintinline{c}{array[i2]}
    \item Write result to \mintinline{c}{sum}.
\end{enumerate}
Due to the non-deterministic nature of context switching, it is possible,
while unlikely, for a context switch to occur from thread 1 to thread 2,
after Step 2. This results in the final write to \mintinline{c}{sum}
being scheduled after thread 2 has finished executing it's operations:
\begin{description}
    \item [1.] Read \mintinline{c}{sum} (\(s\)).
    \item [2.] Read \mintinline{c}{array[i1]} (\(20\)).
    \item [1.] Read \mintinline{c}{sum} (\(s\)).
          \raggedleft
    \item [2.] Read \mintinline{c}{array[i2]} (\(30\)).
    \item [3.] Add \mintinline{c}{sum} and \mintinline{c}{array[i2]} (\(s + 30\)).
    \item [4.] Write result to \mintinline{c}{sum} (\(s_2 \gets s + 30\)).
    \item [3.] Add \mintinline{c}{sum} and \mintinline{c}{array[i1]} (\(s + 20\)).
          \raggedright
    \item [4.] Write result to \mintinline{c}{sum} (\(s_1 \gets s + 20\)).
\end{description}
Observe how this fails to accumulate \mintinline{c}{array[i2]} into
\mintinline{c}{sum} as the value of \mintinline{c}{sum} used by thread
1 is outdated. This problem of synchronisation is known as a
\textbf{race condition}, where the program relies on the
non-deterministic scheduling and execution of threads.

Note that non-deterministic behaviour is not always negative. In some
cases, it is acceptable for there to be more than one equally correct
result, or for there to be non-deterministic components that lead to
the same final result. However, it should be noted that non-determinism
can suggest the existence of a bug in a program, as such behaviour is
often very difficult to debug.
\subsection{Mutual Exclusion}
Consider a program that has a segment of code that must be executed by
several threads, called a \textbf{critical section}, in which threads
may access and update shared memory. To prevent inconsistent accesses
to this shared memory, we can enforce that only one thread may be
inside the critical section at any time. This splits the program in the
following manner:
\begin{itemize}
    \item \textbf{Entry protocol}: This protocol checks if any other
          thread is currently in the critical section. If not, the
          thread is allowed to enter the critical section.
    \item \textbf{Critical section}: This section contains code that
          accesses shared memory.
    \item \textbf{Exit protocol}: This protocol signals that the thread
          has finished executing the critical section.
    \item \textbf{Remainder section}: This section contains code that
          is executed after the critical section, and does not access shared
          memory.
\end{itemize}
In this model, the following properties must also be satisfied:
\begin{itemize}
    \item \textbf{Mutual exclusion} (safety): Only one thread may be in
          the critical section at any time.
    \item \textbf{Progress} (liveness): If no thread is in the critical
          section, and a thread wishes to enter the critical section,
          then the thread will eventually be allowed to enter the
          critical section.
    \item \textbf{Bounded waiting} (fairness): If a thread wishes to
          enter the critical section, then there is a bound on the
          number of times other threads may enter the critical section
          before the thread is allowed to enter.
\end{itemize}
It is important to maintain a queue of waiting threads in a way that
ensures fairness and prevents starvation, where a thread is never
allowed to enter the critical section.

The following sections discuss various mechanisms that can be used to
enforce mutual exclusion.
\subsubsection{Mutex Lock}
A \textbf{mutex} (short for mutual exclusion) is a binary lock that is
either locked or unlocked. A thread that wishes to enter the critical
section must first \textbf{acquire} the lock. If the lock is already
held by another thread, the thread must wait until the lock is
\textbf{released}. When the thread has finished executing the critical
section, it must \textbf{release} the lock so that other threads may
enter the critical section.
\subsubsection{Semaphore}
Semaphores are a generalisation of mutex locks that enforces mutual
exclusion for shared memory that is accessed by multiple threads. A
semaphore is an integer that represents the number of threads that may
access a shared resource. A semaphore first calls the \textbf{wait}
operation, which decrements the semaphore if the semaphore is greater
than 0. Otherwise, the thread is blocked until the semaphore is equal
to zero. This allows a thread to enter the critical section. The thread
then calls the \textbf{signal} operation, which increments the
semaphore, allowing one of the blocked threads to enter the critical
section.
\subsubsection{Monitors}
A \textbf{monitor} is an abstract data type used in object-oriented
programming to enforce mutual exclusion. A monitor encapsulates data
with a set of methods that operate on that data, where each operation
within the monitor is mutually exclusive. In addition to this, monitors
provide \textbf{condition variables} that allow threads to wait for a
certain condition to be met before proceeding. This allows a monitor to
model more complex synchronisation patterns. Condition variables are
equipped with a \textbf{wait} operation that blocks the thread until
another thread signals the condition variable, and a \textbf{signal}
operation that unblocks one of the waiting threads.
\subsubsection{Dekker's Mutual Exclusion Algorithm}
Dekker's algorithm is the first known correct solution to the mutual
exclusion problem where threads communicate through shared memory. A C
implementation of Dekker's algorithm is shown below:
\begin{minted}{cpp}
// Only thread 0 can modify lock[0]
// Only thread 1 can modify lock[1]
bool lock[2] = {false, false}; // initially neither thread has the lock

// whose turn it is to enter the critical section
int turn = 0; // initial value is arbitrary

void T0()
{
    /* Entry protocol */
    lock[0] = true; // thread 0 wants to enter the critical section

    // thread 1 has already requested to enter the critical section
    while (lock[1] == true)
    {
        if (turn == 1) // thread 1 is already in the critical section
        {
            lock[0] = false;  // release the lock
            while (turn == 1) // wait until thread 1 has finished
                ;
            lock[0] = true;   // acquire the lock again
        }
    }

    /* Critical section */

    /* Exit protocol */
    turn = 1;        // give the turn to thread 1
    lock[0] = false; // release the lock

    /* Remainder section */
}

void T1()
{
    /* Entry protocol */
    lock[1] = true; // thread 1 wants to enter the critical section

    // thread 0 has already requested to enter the critical section
    while (lock[0] == true)
    {
        if (turn == 0) // thread 0 is already in the critical section
        {
            lock[1] = false;  // release the lock
            while (turn == 0) // wait until thread 0 has finished
                ;
            lock[1] = true;   // acquire the lock again
        }
    }

    /* Critical section */

    /* Exit protocol */
    turn = 0;        // give the turn to thread 0
    lock[1] = false; // release the lock

    /* Remainder section */
}
\end{minted}
This algorithm guarantees mutual exclusion and prevents deadlocks and
starvation.
\subsection{Memory Consistency Models}
A memory consistency model defines a set of rules that the hardware
follows when operating on memory. These rules allow the programmer to
make assumptions about the order in which memory operations are
executed. A strong memory consistency model (\textbf{sequential
consistency}) is one where:
\begin{itemize}
    \item All memory operations are executed in the order they are
          specified in the program.
    \item Memory operations within a single thread are executed in the
          order they are specified in the program.
\end{itemize}
Here we cannot assume that operations from different threads will
be executed in the relative order they are specified in the program.
If we want the hardware to execute these operations efficiently, we may
choose to relax some of these rules. An example of this may be to allow
program execution to continue while a write operation is being requested.
However, this can violate mutual exclusion if this write operation was
to acquire a lock, and two competing threads try to enter their
respective critical sections. Consider a simplified mutual exclusion
model:
\begin{minted}{c}
void T0()
{
    /* Entry protocol */
    lock[0] = true; // acquire a lock but don't wait for write

    while (lock[1] != true)
        ;

    /* Critical section */

    /* Exit protocol */
    lock[0] = false;
}

void T1()
{
    /* Entry protocol */
    lock[1] = true; // acquire a lock but don't wait for write

    while (lock[0] != true)
        ;

    /* Critical section */

    /* Exit protocol */
    lock[1] = false;
}
\end{minted}
By allowing both threads to execute the condition in the while loop
before the lock has been acquired, we allow both threads to enter the
critical section, violating mutual exclusion.
\subsubsection{Atomic Hardware Operations}
To promote safety, hardware can support atomic operations that perform
writes, comparisons, swaps, etc., to ensure that mutual exclusion is
satisfiable on relaxed memory consistency models. An example of an
atomic hardware operation is shown below:
\begin{minted}{c}
while (!test_and_set(lock))
    ;

/* Critical section */
lock = false;

// requires hardware support to ensure atomic execution
bool test_and_set(bool lock)
{
    if (!lock)
        lock = true;

    return lock;
}
\end{minted}
\subsubsection{Active and Passive Waiting}
When acquiring a lock, a thread could wait for another thread to
release a lock in a number of different ways:
\begin{itemize}
    \item \textbf{Busy wait}: The thread continuously checks if the lock
          has been released. This leads to wasted CPU cycles.
          \begin{minted}{c}
while (!test_and_set(lock))
    ; // spinlock

/* Critical section */
lock = false;
\end{minted}
    \item \textbf{Sleeping}: The thread sleeps for a certain amount of
          time before checking if the lock has been released. This can lead to
          a delay in the thread entering the critical section and also requires
          choosing an appropriate sleep duration.
          \begin{minted}{c}
while (!test_and_set(lock))
    sleep(time);

/* Critical section */
lock = false;
\end{minted}
    \item \textbf{Suspending}: The thread is suspended until the thread
          with the lock releases the lock. This can be done using condition
          variables and can allow multiple threads to wait for a lock to be
          released.
          \begin{minted}{c}
while (!test_and_set(lock))
    event.wait(); // Suspend current thread

/* Critical section */
lock = false;
event.signal(); // Signal other threads
\end{minted}
\end{itemize}
It is crucial to ensure that a thread releases a lock after it has
finished executing it's critical section to prevent threads from blocking
other threads, especially if those threads can raise errors while executing
their critical section.
\subsection{Deadlocks}
A \textbf{deadlock} is a situation where two or more threads are
waiting for each other to release a lock, preventing any of the threads
from progressing. Deadlocks can occur if four conditions hold
simultaneously:
\begin{itemize}
    \item \textbf{Mutual exclusion}: Only one thread may hold a lock at
          any time.
    \item \textbf{Hold and wait}: A thread may hold a lock while waiting
          for another lock.
    \item \textbf{No preemption}: A lock may only be released voluntarily.
    \item \textbf{Circular wait}: A cycle of threads waiting for locks
          exists. For example, thread 0 is waiting for a lock held by
          thread 1, thread 1 is waiting for a lock held by thread 2, and
          thread 2 is waiting for a lock held by thread 0.
\end{itemize}
It is important to be mindful of deadlocks when working with more than
one lock, as they prevent the program from progressing. Note that it may
be possible to prevent deadlocks by ensuring that all locks are available
before a thread enters the critical section, however, this requires
cooperation between threads and may not always be possible. Therefore,
locks are said to be \textbf{non-composable} as they break abstraction
by exposing the internal state of the program, preventing software
components from being developed independently.
\subsubsection{Livelocks and Starvation}
\textbf{Livelocks} are similar situtations to deadlocks, where instead
of being completely blocked, threads are able to perform work, but
unable to make progress. \textbf{Starvation} is a situation where a
thread is unable to enter the critical section because it is never
scheduled to run.
\subsection{Priority Inversion}
\textbf{Priority inversion} is a situation where a low-priority thread
prevents a high-priority thread from executing. Consider the following
scenario:
\begin{enumerate}
    \item A low priority thread acquires a lock for a resource.
    \item A high priority thread waits for the lock to be released.
    \item A medium priority thread that does not require the resource
          is scheduled to run before the low priority thread as it is a
          higher priority.
\end{enumerate}
This situation results in the high priority thread being blocked by both
the low priority thread and the medium priority thread, resulting in a
priority inversion.
\subsection{Thread Safe Libraries}
When calling library functions from different threads, it is important
to ensure that such functions are thread safe, that is, they can be
called from different threads without the risk of incorrect behaviour.
An example of this is a random number generator that relies on a global
seed---it is important to ensure that this seed is correctly updated
when a random number is generated concurrently. A function can be
thread-safe if:
\begin{itemize}
    \item it does not share state across concurrent invocations, or if
    \item it uses mutex locks to protect access to shared state
\end{itemize}
\subsection{Barrier Synchronisation}
A \textbf{barrier} is a synchronisation mechanism that allows threads
to wait for each other to reach a certain point in the program before
continuing. Most parallel programming frameworks provide automatic
barrier synchronisation after a parallel section of code.
\subsection{Ad hoc Synchronisation}
Some operations may require ordering constraints between threads that
must be satisfied for the program to execute correctly. This can be
achieved using a condition variable that signals when a certain
operation has been completed. An example of this is shown below:
\begin{minted}{c}
// Thread 0
operation_A();
A.signal(); // Signal that operation A has been completed
operation_B();

// Thread 1
operation_C();
A.wait(); // Wait for operation A to complete
operation_D(); // Relies on operation A
\end{minted}
In this example, threads 0 and 1 can be executed independently, but
operation D must be executed after operation A has been completed.
\section{Locality}
\subsection{Array Layout}
We can define static arrays in C using the following syntax:
\begin{minted}{c}
int array[N];   // array of N integers
int array[N][M]; // 2D array of N rows and M columns
\end{minted}
In most languages, 2D arrays are stored in row-major order, where the
elements of each row are stored contiguously in memory, one after the
other. This allows for efficient access to elements in the same row as
they are stored in adjacent memory locations. In C, we can access the
element at row \(i\) and column \(j\) of a 2D array using the following
syntax:
\begin{minted}{c}
array[i][j] // *(((int *) array) + i * M + j)
\end{minted}
where \mintinline{c}{M} is the number of columns in the array. If the
length of an array is not known at compile time, we can create a
dynamic array using a heap allocation:
\begin{minted}{c}
int *array = (int *) malloc(sizeof(int) * N);
\end{minted}
This lets us allocate an array of \(N\) integers on the heap. We can
access the element at index \(i\) of the array using the same syntax as
before. Furthermore, we can construct a multi-dimensional array by
allocating memory for each row of the array separately:
\begin{minted}{c}
int **array = (int **) malloc(sizeof(int *) * N);
for (int i = 0; i < N; i++)
{
    array[i] = (int *) malloc(sizeof(int) * M);
}
\end{minted}
This allows us to access the element at row \(i\) and column \(j\) of
the array using the same syntax as before, as the array is defined as a
pointer to a pointer to an integer. If we wanted all elements of the 2d
array to be stored contiguously in memory, we would need to flatten the
array and allocate memory for all elements at once:
\begin{minted}{c}
int *array = (int *) malloc(sizeof(int) * N * M);
\end{minted}
Note that we can no longer access the element at row \(i\) and column
\(j\) of the array using the same syntax as before, as the array is
defined as a pointer to an integer. Instead, we must use:
\begin{minted}{c}
array[i * M + j] // *(array + i * M + j)
\end{minted}
One advantage of creating a flattened array is that it allows for
better cache locality, as all elements are stored contiguously in
memory. On the contrary, multi-dimensional arrays allow arrays to be
jagged, where rows can have a different number of columns.
\subsection{Cache Locality}
Locality of reference is the principle that memory locations that have
been recently accessed are likely to be accessed again in the near
future. There are two types of locality:
\begin{itemize}
    \item \textbf{Temporal locality}: If a memory location is accessed,
          it is likely to be accessed again in the near future.
    \item \textbf{Spatial locality}: If a memory location is accessed,
          memory locations near it are likely to be accessed in the near
          future.
\end{itemize}
When data is loaded into the cache, it is loaded in blocks of memory
called \textbf{cache lines}, typically 128 bytes in size, allowing
adjacent memory locations to be accessed quickly. If we consider a
simple matrix multiplication algorithm:
\begin{minted}{c}
for (int i = 0; i < N; i++)
{
    for (int j = 0; j < N; j++)
    {
        for (int k = 0; k < N; k++)
        {
            C[i][j] = C[i][j] + A[i][k] * B[k][j];
        }
    }
}
\end{minted}
As arrays are stored in row-major order, we have good temporal locality
when accessing elements of \mintinline{text}{C}, given that the indices
\mintinline{text}{i} and \mintinline{text}{j} do not change between
iterations of the innermost loop. Additionally, we have good spatial
locality when accessing elements of \mintinline{text}{A}, as the
innermost loop accesses elements of \mintinline{text}{A} in a
contiguous manner. However, we have poor locality when accessing
elements of \mintinline{text}{B}, as the innermost loop accesses
elements of \mintinline{text}{B} in a non-contiguous manner.
\subsection{False Sharing}
False sharing is a situation where two threads access different data
that belongs to the same cache line. If we consider the above matrix
multiplication example with two threads modifying disjoint columns of
\mintinline{text}{C}, the first thread may load a cache line
corresponding to the first column of \mintinline{text}{C}, which may
contain elements that are accessed by the second thread, even though it
is not accessing those elements. Then, as the second thread modifies
elements of those other columns, the cache line from the first thread
is invalidated as it is shared with another thread. To avoid this
problem, we can align these memory accesses so that they coincide with
cache lines, ensuring that each thread accesses its own cache line.
\subsection{Eliminating ``False'' Dependencies}
In some cases, data dependencies may be eliminated by introducing new
variables without changing the outcome of a program. Consider the
following examples:
\begin{itemize}
    \item \textbf{Flow Dependence}:
          \begin{minted}{c}
f = 42;
...
a = f + 19;
    \end{minted}
          The order of the statements in this example is important, and
          cannot be changed. This is a true dependence.
    \item \textbf{Anti Dependence}:
          \begin{minted}{c}
b = f + 42;
...
f = 42;

// can be rewritten as
b = f + 42;

f2 = 42;
    \end{minted}
          Here any subsequent reads of \mintinline{text}{f} will read
          the value of \mintinline{text}{f2} instead of the original
          value of \mintinline{text}{f}, allowing the statements to be
          reordered.
    \item \textbf{Output Dependence}:
          \begin{minted}{c}
f = 42;
...
f = 19;

// can be rewritten as
f = 42;
...
f2 = 19;
    \end{minted}
          Here the value of \mintinline{text}{f} is not used after the
          second assignment, so the statements can be reordered.
\end{itemize}
We can also eliminate false dependencies in loops. For example, there is
flow dependence in the following loop the first line to the second line:
\begin{minted}{c}
int b;

for (int i = 0; i < N; i++)
{
    b = a[i] + d[i];
    c[i] = b * 2 + (b - 1);
}
\end{minted}
Additionally, as \mintinline{text}{b} is shared between iterations of
the loop, there is an anti-dependence of the second line between
iterations of the loop. These dependencies can be eliminated by using
the following techniques:
\begin{itemize}
    \item \textbf{Thread Local Memory}: Each thread has its own copy of
          \mintinline{text}{b} that is not shared with other threads.
          \begin{minted}{c}
for (int i = 0; i < N; i++)
{
    int b = a[i] + d[i];
    c[i] = b * 2 + (b - 1);
}
    \end{minted}
          We can similarly promote the scalar variable
          \mintinline{text}{b} to an array, however this may consume
          more memory.
    \item \textbf{Creating Copies of Read-Only Variables}: If a variable
          is read-only, we can create multiple copies of the data to be
          shared between threads.
\end{itemize}
\subsection{Memory Access Optimisations}
We can further optimise memory accesses in loops by:
\begin{itemize}
    \item Introducing \textbf{padding} to ensure that data is aligned
          with cache lines. This may be done by increasing the size of
          the data to be a power of 2.
          \begin{minted}{c}
int array[1000][1000];

for (int i = 0; i < 1000; i++)
{
    for (int j = 0; j < 1000; j++)
    {
        array[i][j] = 0;
    }
}

// can be transformed to
int array[1024][1024]; // extend the array dimensions to 1024

for (int i = 0; i < 1024; i++)
{
    for (int j = 0; j < 1024; j++)
    {
        array[i][j] = 0;
    }
}
          \end{minted}
    \item \textbf{Tiling} loops to improve spatial locality. This involves
          breaking the loop into smaller blocks that can be loaded into
          the cache.
          \begin{minted}{c}
for (int x = 0; x < N; x++)
{
    for (int y = 0; y < M; y++)
    {
        S1(x, y);
    }
}

// can be transformed to
for (int x = 0; x < N; x += 32)
{
    for (int y = 0; y < M; y += 32)
    {
        // loop over blocks of size 32x32
        for (int dx = x; dx < x + 32; dx++)
        {
            for (int dy = y; dy < y + 32; dy++)
            {
                S1(dx, dy);
            }
        }
    }
}
\end{minted}
    \item \textbf{Interchanging} loops for coarse grain parallelism or
    better spatial locality.
          \begin{minted}{c}
for (int i = 0; i < N; i++) // spatially local loop
{
    for (int j = 0; j < M; j++) // parallelisable loop
    {
        S1(i, j);
    }
}

// can be transformed to
for (int j = 0; j < M; j++) // coarse grain parallelism
{
    for (int i = 0; i < N; i++) // better spatial locality
    {
        S1(i, j);
    }
}
\end{minted}
    Here the \(j\) loop may be parallelisable, in which case making it
    the outer loop can allow for coarse grain parallelism. Or, the \(i\)
    loop may be iterating over the rows of a matrix, allowing for
    better spatial locality.
    \item \textbf{Loop Fusion} to allow for better temporal locality.
          \begin{minted}{c}
for (int i = 0; i < N; i++)
{
    S1(i);
}

for (int i = 0; i < N; i++)
{
    S2(i); // may use data from S1
}

// can be transformed to
for (int i = 0; i < N; i++)
{
    S1(i);
    S2(i); // data access is localised from previous operation
}
\end{minted}
    \item \textbf{Loop Distribution} to allow potential parallelism.
          \begin{minted}{c}
for (int i = 0; i < N; i++)
{
    S1(i); // can be parallelised
    S2(i); // cannot be parallelised
}

// can be transformed to
for (int i = 0; i < N; i++) // parallelisable loop
{
    S1(i);
}

for (int i = 0; i < N; i++) // non-parallelisable loop
{
    S2(i);
}
\end{minted}
    \item \textbf{Loop Peeling} may be used to remove parts of a loop
    that are more expensive than others.
          \begin{minted}{c}
for (int i = 0; i < N; i++)
{
    S1(i); // S1(0) is more expensive than remaining iterations
           // or may be read by all subsequent iterations
}

// can be transformed to
S1(0); // extract the first iteration
for (int i = 1; i < N; i++) // may parallelise remaining iterations
{
    S1(i);
}
\end{minted}
    \item \textbf{Loop Unrolling} may be used to remove loops entirely
    by manually unrolling them. This may be useful for small loops.
    Here we avoid the overhead of the loop counter and can more finely
    control which iterations are parallelised.
          \begin{minted}{c}
for (int i = 0; i < N; i++)
{
    S1(i);
}

// can be transformed to
S1(0);
S1(1);
...
S1(N - 1);
\end{minted}
\end{itemize}
\subsection{Barriers for Parallelisation Speedup}
To overcome barries when parallelising code, we can consider the
following:
\begin{itemize}
    \item If the overhead of parallelisation is too high:
        \begin{itemize}
            \item look for more coarse-grained parallelism
        \end{itemize}
    \item If some threads are doing more work than others:
        \begin{itemize}
            \item use dynamic work allocation to balance the work across cores
            \item overdecompose the work into many work units by creating
            more threads than cores to ensure that all cores are kept busy
        \end{itemize}
    \item If threads are waiting for each other:
        \begin{itemize}
            \item lock at different granularities
        \end{itemize}
    \item Apply loop transformations to expose parallelism
    \item Alter data structures to improve data locality
\end{itemize}
\end{document}
